\documentclass[a4paper, 12pt]{article}
\usepackage{fullpage}

\usepackage{amsmath}
\usepackage{amssymb}
\usepackage{amsfonts}
\usepackage{amsthm}
\usepackage{mathtools}
\usepackage{soul}

\input{python.tex}

\usepackage{xepersian}

\settextfont{XB Zar}

\theoremstyle{definition}
\newtheorem{solution}{پاسخ}[section]
\newtheorem{solutionsection}{مورد}[solution]
\renewcommand{\thesolution}{\arabic{solution}}

\renewcommand{\baselinestretch}{1.3}

\definecolor{OliveGreen}{rgb}{0.84,1,0.64}

\begin{document}

\textbf{حسابگری زیستی؛}

\textbf{حل مسئله فروشنده دوره‌گرد به کمک الگوریتم ژنتیک؛}

\textbf{کامیار میرزاوزیری؛ 610396152}

\hrulefill

\section{خواندن مسئله}

پیش از هر کاری لازم است مسئله را از فایل ورودی بخوانیم و آن را پارس کنیم تا بتوانیم به تعداد و فواصل شهرها (رئوس) دسترسی داشته باشیم. برای این منظور کلاسی به نام
\texttt{TSP}
تعریف می‌کنیم که هر
\texttt{instance}
آن یک مسئله فروشندهٔ دوره‌گرد می‌باشد و با گرفتن مسیر فایل ورودی ایجاد می‌شود.

\LTR
\begin{lstlisting}[language=Python]
class TSP:
    EDGE_WEIGHT_TYPE_EXPLICIT = 'EXPLICIT'
    EDGE_WEIGHT_TYPE_GEO = 'GEO'
    EDGE_WEIGHT_TYPE_EUC_2D = 'EUC_2D'

    def __init__(self, path: str):
        self.dimension = None
        self.edge_with_type = None
        self.distances = None

        with open(path) as f:
            lines = f.readlines()
        lines = list(map(lambda line: line.strip(), lines))
        self.__parse(lines)
\end{lstlisting}
\RTL

برای جلوگیری از طولانی شدن توابع مربوط به
\texttt{parse}
را در این گزارش نمی‌آوریم اما در فایل سورس پایتون موجود است.

\section{کروموزوم‌ها و جمعیت اولیه}

اولین مرحله در حل یک مسئله به کمک الگوریتم ژنتیک تعریف کروموزوم یا همان نمایش یک پاسخ برای مسئله است. برای مسئله فروشندهٔ دوره‌گرد به دنبال یک دور در گراف هستیم که می‌توانیم این دور را با دنباله‌ای از رئوس نشان دهیم که عضو آخر دنباله به عضو اول متصل است. برای مثال کروموزوم
$1, 3, 2$
دوری را نشان می‌دهد که از
$1$
شروع شده، به
$2$
می‌رود، سپس به
$3$
می‌رود و نهایتا به
$1$
باز می‌گردد. لذا کروموزوم‌ها به صورت جایگشت‌هایی از رئوس می‌باشند. پس جایگشت‌ها را به عنوان کروموزوم انتخاب می‌کنیم. در پیاده‌سازی پایتون کلاس زیر را تعریف می‌کنیم.

\LTR
\begin{lstlisting}[language=Python]
class Chromosome:
    def __init__(self):
        global tsp
        self.__data = list(range(tsp.dimension))
        random.shuffle(self.__data)
        self.__cached_cost = 0
        self.__cache_is_valid = False
\end{lstlisting}
\RTL

در
\texttt{constructor}
این کلاس یک جایگشت رندوم ایجاد می‌کنیم، و جمعیت اولیه را به کمک این متد ایجاد می‌کنیم.

\section{تابع برازش}

در مرحلهٔ بعد باید تابع برازش را برای مسئله تعریف کنیم یا به عبارتی دیگر بدانیم کدام کروموزوم برازندگی بیشتری دارد. به سادگی می‌توان متوجه شد که به دنبال کمینه کردن طول دور هستیم پس تابع هزینه می‌تواند طول دور باشد و تابع برازش نیز برعکس تابع هزینه باشد. به عبارت دیگر هرچه هزینهٔ یک کروموزوم کمتر باشد آن را برازنده‌تر می‌دانیم. پس عملگر کوچک‌تر را به کلاس
\texttt{Chromosome}
اضافه می‌کنیم.

\LTR
\begin{lstlisting}[language=Python]
def __lt__(self, other):
    return self.cost() > other.cost()
    
def cost(self):
    global tsp
    if not self.__cache_is_valid:
        self.__cached_cost = 0
        for i in range(len(self.__data)):
            self.__cached_cost += tsp.distances[self.__data[i - 1]][self.__data[i]]
        self.__cache_is_valid = True
    return self.__cached_cost
\end{lstlisting}
\RTL

با توجه به تعریف بالا
\lr{\texttt{c1 < c2}}
به این معنی است که هزینهٔ
\texttt{c1}
از
\texttt{c2}
بیشتر است یا به عبارت دیگر
\texttt{c2}
برازنده‌تر است.

برای نمایش کروموزوم‌ها متد زیر را به کلاس
\texttt{Chromosome}
اضافه می‌کنیم.

\LTR
\begin{lstlisting}[language=Python]
def __str__(self):
    return self.__data.__str__() + ': ' + str(self.cost())
\end{lstlisting}
\RTL

حال برای مثال دو کروموزوم تصادفی ایجاد می‌کنیم و آن را نمایش می‌دهیم، همچنین بررسی می‌کنیم که کدام کروموزوم برازنده‌تر است.

\LTR
\begin{lstlisting}[language=Python]
BAYG29 = 'testcase.bayg29.tsp'
tsp = TSP(BAYG29)
c1 = Chromosome()
c2 = Chromosome()
print(c1)
print(c2)
print(c1 < c2)
\end{lstlisting}
\RTL

حاصل زیر چاپ می‌شود.

\LTR
\begin{lstlisting}[language=Python]
[9, 7, 23, 3, 11, 14, 25, 4, 13, 12, 27, 17, 24, 28, 20, 19, 16, 18, 0, 5, 26, 6, 15, 22, 8, 21, 1, 10, 2]: 4920
[19, 6, 22, 17, 25, 2, 20, 12, 15, 14, 23, 10, 11, 27, 18, 26, 24, 28, 9, 4, 21, 1, 8, 13, 0, 16, 5, 7, 3]: 4980
False    
\end{lstlisting}
\RTL

همانطور که دیده می‌شود هزینهٔ
\texttt{c1}
کمتر از
\texttt{c2}
است لذا برازنده‌تر است پس
\lr{\texttt{c1 < c2}}
نادرست تعبیر می‌شود.

\section{انتخاب}
برای انتخاب از ترکیب روش‌های انتخاب رتبه‌ای و رولت چندنشانه‌گر استفاده می‌کنیم. به این صورت که ابتدا جامعه را رتبه‌بندی می‌کنیم و سپس از رولت با
$N$
اشاره‌گر استفاده می‌کنیم. برای این منظور و همچنین کاربردهایی که در ادامه می‌بینیم، کلاس
\texttt{Population}
را پیاده‌سازی می‌کنیم.

\LTR
\begin{lstlisting}[language=Python]
class Population(list):
    def __init__(self):
        if type(countOrData) == int:
            self.__data = [Chromosome() for _ in range(countOrData)]
        elif type(countOrData) == list:
            self.__data = countOrData
        else:
            raise Exception()
        self.__data.sort()

    def __choose(self):
        n = len(self.__data)
        roulette = sum([[i] * (i + 1) for i in range(n)], [])
        turning = random.randint(0, n)
        roulette = roulette[turning:] + roulette[:turning]
        pointers = range(0, len(roulette), math.ceil(len(roulette) / n))

        choices = []
        for pointer in pointers:
            choices.append(self.__data[roulette[pointer]])

        return choices
\end{lstlisting}
\RTL


\section{بازترکیب}
حال که تکلیف جمعیت اولیه و تابع برازش مشخص شد باید برای
\texttt{crossover}
تصمیم بگیریم. به طور شهودی به نظر می‌رسد که بازترکیب ترتیبی یا
\lr{\texttt{Order Recombination}}
راه مناسبی برای ترکیب دو کروموزوم باشد. برای پیاده‌سازی بازترکیب از عملگر ضرب بین دو کروموزوم استفاده می‌کنیم.

\LTR
\begin{lstlisting}[language=Python]
def __mul__(self, other):
    global tsp
    (side1, side2) = random.sample(range(tsp.dimension + 1), 2)

    start = min(side1, side2)
    end = max(side1, side2)
    if PRINT_SLICE_INFO:
        print(start, end)

    first_child = Chromosome()
    first_child.__data = self.__crossover(self.__data, other.__data, start, end)

    second_child = Chromosome()
    second_child.__data = self.__crossover(other.__data, self.__data, start, end)

    return [first_child, second_child]

@staticmethod
def __crossover(mother_data: list, father_data: list, start: int, end: int):
    dimension = len(mother_data)
    data = [None] * dimension
    data[start:end] = mother_data[start:end]
    i = end
    for v in father_data[end:] + father_data[:end]:
        if v not in data:
            if i == start:
                i = end
            if i == dimension:
                i = 0
            data[i] = v
            i += 1
    return data
\end{lstlisting}
\RTL

حال برای مثال بین دو کروموزوم تصادفی عمل ضرب انجام می‌دهیم و دو فرزند به وجود آمده را بررسی می‌کنیم.

\LTR
\begin{lstlisting}[language=Python]
BAYG29 = 'testcase.bayg29.tsp'
tsp = TSP(BAYG29)
c1 = Chromosome()
c2 = Chromosome()
print(c1)
print(c2)
(c3, c4) = c1 * c2
print(c3)
print(c4)
\end{lstlisting}
\RTL

حاصل زیر چاپ می‌شود.

\LTR
\begin{lstlisting}[language=Python]
[
    0, 8, 11, 17, 24, 7, 13, 15, 23,
    (*@\sethlcolor{OliveGreen}\hl{5, 9, 4, 10, 12, 27, 16, 6, 14,}@*)
    3, 21, 1, 22, 28, 20, 19, 18, 2, 25, 26
]: 4677
[
    0, 8, 3, 2, 18, 5, 28, 15, 12,
    (*@\sethlcolor{yellow}\hl{\mbox{9, 11, 21, 4, 10, 23, 27, 14, 16,}}@*)
    13, 17, 6, 19, 7, 26, 1, 24, 22, 20, 25
]: 4758
9 18
[
    8, 3, 2, 18, 28, 15, 11, 21, 23,
    (*@\sethlcolor{OliveGreen}\hl{5, 9, 4, 10, 12, 27, 16, 6, 14,}@*)
    13, 17, 19, 7, 26, 1, 24, 22, 20, 25, 0
]: 4953
[
    8, 17, 24, 7, 13, 15, 5, 12, 6,
    (*@\sethlcolor{yellow}\hl{\mbox{9, 11, 21, 4, 10, 23, 27, 14, 16,}}@*)
    3, 1, 22, 28, 20, 19, 18, 2, 25, 26, 0
]: 5093    
\end{lstlisting}
\RTL

\section{جهش}
پس از تعریف
\texttt{crossover}
نوبت به تعریف جهش یا همان
\texttt{mutation}
می‌رسد. شهودا به نظر می‌رسد جهش درجی یا
\lr{\texttt{Insert Mutation}}
در این مسئله بهتر از جهش جابجایی عمل کند لذا از این جهش استفاده می‌کنیم. برای پیاده‌سازی جهش از عملگر معکوس یعنی
\texttt{\textasciitilde}
استفاده می‌کنیم.

\LTR
\begin{lstlisting}[language=Python]
def __invert__(self):
    global tsp

    result = Chromosome()
    result.__data = self.__data

    count = random.randint(0, MUTATION_DEGREE)
    for _ in range(count):
        (src, dst) = random.sample(range(tsp.dimension), 2)
        if PRINT_SLICE_INFO:
            print(src, dst)
        v = result.__data[src]
        result.__data = result.__data[:src] + result.__data[src + 1:]
        result.__data = result.__data[:dst] + [v] + result.__data[dst:]

    return result
\end{lstlisting}
\RTL

حال برای مثال یک کروموزوم تصادفی ایجاد می‌کنیم و عمل جهش را روی آن انجام می‌دهیم. سپس نتیجه را بررسی می‌کنیم.

\LTR
\begin{lstlisting}[language=Python]
BAYG29 = 'testcase.bayg29.tsp'
tsp = TSP(BAYG29)
c = Chromosome()
print(c)
print(~c)    
\end{lstlisting}
\RTL
حاصل زیر چاپ می‌شود.

\LTR
\begin{lstlisting}[language=Python]
    [19, 21, 24, 26, 25, 14, 15, 3, 27, 13, 17, 10, 2, 7, (*@\sethlcolor{OliveGreen}\hl{23}@*), 9, 4, 22, 0, 8, 6, 16, 12, 28, 11, 20, 5, 18, 1]: 4649
    14 5
    [19, 21, 24, 26, 25, (*@\sethlcolor{OliveGreen}\hl{23}@*), 14, 15, 3, 27, 13, 17, 10, 2, 7, 9, 4, 22, 0, 8, 6, 16, 12, 28, 11, 20, 5, 18, 1]: 4738  
\end{lstlisting}
\RTL

توجه کنیم که در جهشی که پیاده‌سازی کردیم می‌تواند به تعدادی تصادفی جهش درجی رخ دهد. که در این مثال برای سادگی این تعداد را یک در نظر گرفتیم.

\section{جایگزینی}
به کمک کدهایی که در بالا نوشتیم می‌توانیم نسل بعدی را از روی نسل قبلی ایجاد کنیم. حال نیاز داریم تصمیم بگیریم که کدام کروموزوم‌ها را نگه داریم و کدام‌ها را حذف کنیم. تصمیم می‌گیریم
$r_0$
درصد برتر از فرزندان نسل جدید،
$r_1$
درصد از سایر فرزندان نسل جدید،
$r_2$
درصد از سایر جامعه قبلی و
$100 - r_0 - r_1 - r_2$
درصد برتر از جامعه قبلی را به عنوان جمعیت جدید در نظر بگیریم.

\LTR
\begin{lstlisting}[language=Python]
def __replacement(self, children):
    n = len(children.__data)
    best_children_count = math.floor(REPLACEMENT[0] * n)
    other_children_count = math.floor(REPLACEMENT[1] * n)
    other_parents_count = math.floor(REPLACEMENT[2] * n)
    best_parents_count = n - best_children_count - other_children_count - other_parents_count
    self.__data = (
        children.__data[-best_children_count:] +
        random.sample(children.__data[:(n - best_children_count)], other_children_count) +
        random.sample(self.__data[:(n - best_parents_count)], other_parents_count) +
        self.__data[-best_parents_count:]
    )
    self.__data.sort()
\end{lstlisting}
\RTL

\section{شرط توقف}
شرط توقف را به این صورت در نظر می‌گیریم که بعد از تعداد مشخصی نسل هر بار تغییرات هر نسل نسبت به نسل قبل از خود کمتر از مقدار مشخصی باشد. برای این منظور دو مقدار مشخص زیر را تعریف می‌کنیم.

\LTR
\begin{lstlisting}[language=Python]
IMPROVEMENT_THRESHOLD
STAGNANCY_THRESHOLD
\end{lstlisting}
\RTL

شرط پایان را نیز به صورت زیر تعریف می‌کنیم.

\LTR
\begin{lstlisting}[language=Python]
if improvement < IMPROVEMENT_THRESHOLD:
    stagnancy += 1
    if stagnancy == math.floor(STAGNANCY_THRESHOLD / 2):
        MUTATION_DEGREE = math.floor(MUTATION_DEGREE * STAGNANCY_ESCAPE_DEGREE)
    if stagnancy >= math.floor(STAGNANCY_THRESHOLD / 2):
        population.escape_stagnancy(STAGNANCY_ESCAPE_PROPORTION)
    if stagnancy >= STAGNANCY_THRESHOLD:
        break
else:
    stagnancy = 0
\end{lstlisting}
\RTL

توجه کنید که برای اجرا شدن کد بالا لازم است
\texttt{answer()}
را روی
\texttt{Population}
تعریف کنیم.

\LTR
\begin{lstlisting}[language=Python]
    def answer(self) -> Chromosome:
        return self.__data[-1]
\end{lstlisting}
\RTL

همچنین لازم است تفاوت دو کروموزوم را بتوانیم اندازه‌گیری کنیم که این کار را با تعریف عمل تفریق روی کلاس
\texttt{Chromosome}
انجام می‌دهیم.

\LTR
\begin{lstlisting}[language=Python]
def __sub__(self, other):
    return other.cost() - self.cost()
\end{lstlisting}
\RTL

نکته قابل توجه و خلاقانه در این قسمت این است که هنگامی که به نیمی از آستانه رکود می‌رسیم، از متدهایی برای فرار از رکود و پیش‌گیری توقف استفاده می‌کنیم. برای فرار از رکود لازم است به نوعی تنوع را در جامعه زیاد کنیم و از همگرایی زودرس جلوگیری کنیم. به این منظور اولا، حداکثر تعداد جهش‌های درجی که روی یک کروموزوم می‌تواند انجام شود را در پارامتر خاصی ضرب می‌کنیم، و همچنین درصدی از پایین جامعه را حذف و با کروموزوم‌های تصادفی جایگزین می‌کنیم تا از رکود جلوگیری شود. این راهکارهای ابتکاری با توجه به مشاهدات انجام شده توانست تأثیر به‌سزایی در اجرای الگوریتم بگذارد.

\LTR
\begin{lstlisting}[language=Python]
def escape_stagnancy(self, proportion: float):
    count = math.floor(len(self.__data) * proportion)
    self.__data[:count] = [Chromosome() for _ in range(count)]
    self.__data.sort()
\end{lstlisting}
\RTL

\section{کد نهایی}
در نهایت کد زیر را برای حل سؤال می‌نویسیم.

\LTR
\begin{lstlisting}[language=Python]
population = Population(N)
answer = population.answer()
stagnancy = 0
i = 0
while True:
    population.iterate()
    improvement = population.answer() - answer
    answer = population.answer()

    if VERBOSE_LEVEL > -1:
        print(f"Iteration: {i}")
    if VERBOSE_LEVEL > 0:
        print(f"Best Answer: {population.answer()}")
    elif VERBOSE_LEVEL == 0:
        print(f"Best Answer: {population.answer().cost()}")
    if VERBOSE_LEVEL > 1:
        print(f"All Answers: {population.answers()}")

    if improvement < IMPROVEMENT_THRESHOLD:
        stagnancy += 1
        if stagnancy >= STAGNANCY_THRESHOLD:
            break
    else:
        stagnancy = 0

    i += 1

if VERBOSE_LEVEL == 0:
    print(population.answer())    
\end{lstlisting}
\RTL

پارامترهای الگوریتم را با آزمون و خطا به دست می‌آوریم. به نظر می‌رسد هرچه جمعیت بزرگ‌تر باشد الگوریتم پاسخ بهتری می‌دهد چرا که دیرتر همگرا می‌شود. از طرفی آستانهٔ رکود و بهبودی در حد کم کافی به نظر می‌رسد و زیاد کردن آن تفاوت چندانی در عملکرد ندارد.

برای جلوگیری از همگرایی زودرس باید احتمال جهش را بسیار بالا و سهم فرزندان در هر نسل را پایین در نظر بگیریم.

همچنین باید تابع
\texttt{iterate}
و توابع دیگر که برا اجرای کد فوق لازم می‌باشند را به
\texttt{Population}
اضافه کنیم.

\LTR
\begin{lstlisting}[language=Python]
def iterate(self):
    children = self.__crossover()
    children.__mutate()
    self.__replacement(children)

def __crossover(self):
    parents = self.__choose()
    random.shuffle(parents)
    children = []
    for i in range(0, len(parents) - 1, 2):
        children += parents[i] * parents[i + 1]
    return Population(None, children)

def __mutate(self):
    for child in self.__data:
        child = ~child

def answers(self) -> list:
    return list(map(lambda c: c.cost(), self.__data))
\end{lstlisting}
\RTL

\section{نتایج}

حال می‌توانیم کد موجود را روی داده‌ها اجرا کنیم و نتایج را بررسی کنیم.

\subsection{\lr{bayg29}}

پارامترها را به صورت زیر مقداردهی می‌کنیم.

\LTR
\begin{lstlisting}[language=Python]
N = 240
MUTATION_DEGREE = 9
REPLACEMENT = [.1, .4, .4]
STAGNANCY_ESCAPE_DEGREE = 2
STAGNANCY_ESCAPE_PROPORTION = 0.9
IMPROVEMENT_THRESHOLD = 1
STAGNANCY_THRESHOLD = 40
\end{lstlisting}
\RTL


اجرای الگوریتم پس از
\texttt{133}
نسل متوقف می‌شود و پاسخ
\texttt{1620}
را می‌یابد.

\LTR
\begin{lstlisting}[language=Python]
Iteration: 133
Best Answer: [19, 1, 20, 4, 28, 2, 25, 8, 11, 5, 27, 0, 12, 15, 23, 7, 26, 22, 6, 24, 18, 10, 21, 16, 13, 17, 14, 3, 9]: 1620
\end{lstlisting}
\RTL

\subsection{\lr{gr229}}
برای پاسخ به این مسئله و مسئلهٔ بعدی که حجم بزرگ‌تری دارند لازم است الگوریتم را سریع‌تر کنیم. با یک بررسی ساده و اندازه‌گیری مراحل مختلف متوجه می‌شویم بیشترین زمان صرف
\texttt{crossover}
می‌شود که خوشبختانه به راحتی می‌توانیم آن را به صورت موازی انجام دهیم.

\LTR
\begin{lstlisting}[language=Python]
def __crossover(self):
    parents = self.__choose()
    random.shuffle(parents)

    P_COUNT = os.cpu_count()

    def pair_chunk_calculator(i, pair_chunk, rd):
        rd[i] = (sum([pair[0] * pair[1] for pair in pair_chunk], []))

    pair_chunks = numpy.array_split([[parents[i], parents[i + 1]] for i in range(0, len(parents) - 1, 2)], P_COUNT)
    manager = mp.Manager()
    rd = manager.dict()
    processes = [mp.Process(
        target=pair_chunk_calculator,
        args=(i, pair_chunks[i], rd)
    ) for i in range(P_COUNT)]

    for p in processes:
        p.start()

    for p in processes:
        p.join()

    return Population(sum(rd.values(), []))
\end{lstlisting}
\RTL

حال می‌توانیم با سرعت بسیار بیشتری این پردازش را به صورت موازی پیش ببریم.


پارامترها را به صورت زیر مقداردهی می‌کنیم.

\LTR
\begin{lstlisting}[language=Python]
N = 240
MUTATION_DEGREE = 9
REPLACEMENT = [.1, .4, .4]
STAGNANCY_ESCAPE_DEGREE = 2
STAGNANCY_ESCAPE_PROPORTION = 0.9
IMPROVEMENT_THRESHOLD = 1
STAGNANCY_THRESHOLD = 40
\end{lstlisting}
\RTL

اجرای الگوریتم پس از
\texttt{1088}
نسل متوقف می‌شود و پاسخ
\texttt{2353}
را می‌یابد.

\LTR
\begin{lstlisting}[language=Python]
Iteration: 1088
Best Answer: [127, 124, 123, 97, 114, 117, 120, 121, 118, 106, 96, 94, 95, 98, 73, 89, 88, 87, 86, 80, 78, 76, 72, 71, 70, 69, 57, 15, 16, 11, 1, 25, 21, 12, 0, 2, 3, 4, 5, 9, 52, 50, 223, 224, 217, 216, 226, 225, 227, 228, 51, 58, 53, 54, 56, 82, 83, 81, 105, 111, 93, 102, 101, 100, 99, 92, 91, 24, 19, 18, 10, 8, 6, 7, 17, 55, 22, 23, 84, 85, 79, 62, 64, 65, 63, 61, 59, 60, 66, 67, 68, 77, 75, 74, 90, 34, 32, 33, 103, 107, 109, 113, 128, 119, 115, 112, 125, 130, 168, 166, 167, 176, 178, 39, 38, 40, 180, 185, 184, 182, 162, 134, 135, 136, 137, 138, 152, 153, 154, 148, 147, 146, 145, 151, 150, 149, 204, 210, 202, 157, 200, 156, 144, 143, 142, 140, 139, 141, 122, 116, 126, 129, 131, 110, 108, 104, 36, 35, 20, 14, 13, 26, 27, 30, 29, 28, 31, 37, 165, 132, 133, 161, 177, 183, 46, 44, 189, 188, 187, 193, 191, 198, 172, 171, 163, 160, 159, 155, 158, 203, 211, 205, 206, 219, 209, 208, 207, 215, 214, 213, 212, 218, 220, 222, 221, 201, 199, 194, 195, 186, 49, 47, 45, 41, 42, 43, 48, 190, 192, 196, 197, 164, 175, 174, 173, 181, 179, 169, 170]: 2353
\end{lstlisting}
\RTL

\subsection{\lr{pr1002}}


پارامترها را به صورت زیر مقداردهی می‌کنیم.

\LTR
\begin{lstlisting}[language=Python]
N = 300
MUTATION_DEGREE = 9
REPLACEMENT = [.1, .4, .4]
STAGNANCY_ESCAPE_DEGREE = 2
STAGNANCY_ESCAPE_PROPORTION = 0.9
IMPROVEMENT_THRESHOLD = 1_000
STAGNANCY_THRESHOLD = 8
\end{lstlisting}
\RTL


اجرای الگوریتم پس از
\texttt{247}
نسل متوقف می‌شود و پاسخ
\texttt{3996150}
را می‌یابد.

\LTR
\begin{lstlisting}[language=Python]
Iteration: 247
Best Answer: [
    700, 988, 754, 660, 737, 676, 670, 790, 997, 160, 559, 583, 220, 832,
    689, 677, 427, 324, 303, 128, 345, 762, 925, 663, 640, 555, 447, 576,
    554, 713, 619, 376, 355, 44, 127, 297, 359, 386, 628, 207, 178, 144,
    244, 558, 548, 870, 864, 991, 222, 391, 104, 296, 516, 494, 174, 138,
    243, 569, 705, 749, 586, 282, 871, 973, 744, 431, 423, 993, 302, 43,
    37, 11, 570, 485, 567, 778, 928, 751, 587, 239, 245, 0, 167, 820,
    182, 199, 824, 813, 844, 858, 878, 806, 572, 286, 413, 333, 25, 14,
    597, 638, 568, 954, 903, 794, 471, 461, 265, 818, 389, 994, 52, 742,
    468, 719, 963, 946, 780, 378, 945, 936, 704, 432, 601, 743, 659, 776,
    690, 784, 992, 771, 456, 197, 484, 478, 733, 711, 933, 850, 834, 593,
    981, 615, 758, 917, 797, 594, 337, 644, 836, 912, 739, 655, 748, 48,
    354, 523, 854, 828, 1000, 606, 173, 161, 216, 15, 403, 102, 374, 368,
    649, 340, 783, 753, 283, 371, 147, 443, 487, 517, 956, 356, 65, 255,
    32, 990, 860, 941, 763, 696, 938, 915, 228, 211, 189, 157, 272, 463,
    652, 907, 213, 486, 519, 187, 241, 120, 592, 508, 261, 496, 514, 550,
    561, 268, 231, 141, 95, 135, 108, 257, 566, 35, 351, 698, 126, 452,
    880, 610, 505, 185, 346, 29, 23, 318, 294, 152, 464, 430, 861, 849,
    775, 887, 405, 177, 162, 4, 2, 46, 39, 599, 799, 675, 654, 402,
    395, 725, 947, 959, 580, 205, 208, 829, 816, 811, 489, 101, 92, 73,
    76, 133, 504, 975, 863, 625, 740, 875, 934, 709, 662, 658, 396, 290,
    59, 19, 293, 349, 38, 454, 411, 419, 407, 699, 117, 130, 473, 843,
    913, 842, 998, 506, 520, 563, 480, 299, 275, 458, 650, 636, 172, 63,
    287, 417, 380, 814, 759, 908, 596, 590, 920, 960, 931, 627, 620, 603,
    477, 886, 591, 530, 276, 312, 64, 308, 165, 212, 770, 767, 618, 490,
    564, 789, 761, 694, 716, 428, 536, 479, 158, 251, 166, 278, 450, 935,
    951, 726, 944, 787, 755, 273, 466, 70, 96, 106, 204, 540, 13, 383,
    795, 792, 129, 113, 87, 51, 201, 175, 159, 119, 651, 741, 465, 353,
    788, 680, 358, 962, 950, 924, 702, 687, 472, 442, 515, 560, 524, 551,
    873, 722, 579, 285, 89, 226, 192, 56, 394, 684, 766, 624, 315, 79,
    237, 225, 292, 323, 420, 71, 710, 961, 793, 622, 451, 331, 140, 277,
    661, 321, 614, 311, 154, 264, 898, 470, 475, 256, 503, 453, 577, 246,
    377, 634, 791, 686, 695, 674, 721, 890, 821, 807, 643, 656, 341, 425,
    41, 234, 66, 36, 888, 868, 852, 896, 892, 635, 90, 259, 926, 966,
    798, 444, 334, 968, 845, 511, 756, 57, 31, 295, 325, 557, 229, 269,
    193, 215, 217, 100, 328, 607, 728, 985, 382, 481, 549, 556, 510, 547,
    876, 840, 918, 929, 785, 708, 693, 669, 720, 715, 642, 344, 859, 553,
    909, 545, 529, 267, 202, 562, 581, 439, 441, 629, 288, 367, 440, 429,
    717, 573, 571, 238, 248, 872, 964, 802, 809, 613, 865, 808, 804, 769,
    980, 972, 703, 727, 667, 250, 139, 75, 82, 317, 681, 692, 989, 322,
    67, 313, 47, 103, 348, 927, 969, 701, 948, 841, 891, 851, 855, 831,
    957, 404, 114, 109, 26, 1, 242, 145, 459, 194, 8, 284, 94, 300,
    535, 314, 513, 819, 399, 360, 171, 179, 565, 902, 921, 846, 866, 877,
    916, 835, 280, 500, 538, 434, 899, 882, 543, 526, 236, 203, 270, 93,
    457, 879, 731, 874, 955, 952, 847, 901, 848, 839, 532, 137, 316, 600,
    774, 949, 786, 352, 83, 77, 9, 111, 326, 488, 582, 379, 61, 69,
    291, 330, 22, 97, 773, 1001, 539, 781, 923, 884, 528, 900, 668, 967,
    986, 723, 653, 697, 803, 678, 747, 772, 919, 626, 156, 118, 262, 546,
    768, 970, 664, 339, 281, 27, 632, 996, 437, 460, 306, 7, 602, 416,
    397, 393, 357, 33, 525, 151, 146, 426, 28, 301, 143, 375, 630, 148,
    304, 319, 289, 332, 134, 385, 49, 34, 80, 729, 685, 932, 800, 534,
    448, 263, 122, 412, 415, 329, 170, 125, 110, 683, 976, 738, 688, 665,
    735, 679, 123, 455, 574, 365, 645, 521, 531, 853, 910, 712, 971, 88,
    115, 497, 498, 209, 219, 258, 195, 235, 240, 398, 414, 168, 221, 271,
    641, 724, 779, 502, 482, 499, 132, 91, 105, 17, 42, 274, 418, 492,
    116, 252, 150, 99, 646, 507, 483, 522, 830, 232, 214, 223, 74, 469,
    249, 612, 409, 279, 16, 364, 889, 837, 826, 552, 922, 433, 764, 965,
    930, 604, 765, 897, 867, 777, 883, 584, 589, 943, 978, 760, 914, 895,
    801, 639, 939, 730, 856, 107, 343, 942, 390, 633, 436, 392, 647, 757,
    805, 823, 298, 163, 501, 218, 45, 198, 183, 200, 616, 362, 608, 384,
    363, 335, 86, 336, 691, 62, 58, 50, 462, 186, 260, 98, 81, 347,
    369, 575, 578, 493, 169, 153, 491, 937, 953, 474, 885, 881, 822, 815,
    982, 387, 810, 595, 585, 611, 648, 621, 361, 342, 940, 827, 812, 527,
    893, 906, 833, 68, 10, 12, 401, 112, 372, 637, 388, 350, 410, 18,
    979, 750, 435, 732, 734, 495, 424, 230, 210, 544, 476, 911, 512, 542,
    533, 894, 838, 825, 537, 857, 180, 320, 6, 987, 983, 605, 623, 745,
    438, 366, 84, 196, 227, 233, 266, 310, 136, 78, 55, 254, 124, 24,
    305, 671, 672, 999, 974, 746, 657, 307, 406, 408, 752, 682, 673, 617,
    707, 588, 467, 191, 20, 53, 54, 381, 714, 706, 373, 631, 188, 155,
    164, 176, 181, 40, 72, 247, 862, 736, 224, 309, 5, 400, 609, 422,
    509, 541, 782, 904, 905, 518, 598, 817, 869, 30, 449, 85, 184, 149,
    131, 121, 445, 796, 718, 338, 190, 206, 995, 142, 666, 984, 421, 253,
    977, 446, 958, 60, 21, 327, 3, 370
]: 3996150
\end{lstlisting}
\RTL

\end{document}


\LTR
\begin{lstlisting}[language=Python]
\end{lstlisting}
\RTL
